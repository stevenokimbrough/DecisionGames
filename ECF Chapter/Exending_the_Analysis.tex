\def\figtop{\rule{\textwidth}{0.5mm}}
\def\figbot{\rule{\textwidth}{0.5mm}}

\documentclass[11pt, oneside]{article}   	% use "amsart" instead of "article" for AMSLaTeX format
\usepackage{geometry}                		% See geometry.pdf to learn the layout options. There are lots.
\geometry{letterpaper}                   		% ... or a4paper or a5paper or ... 
%\geometry{landscape}                		% Activate for rotated page geometry
%\usepackage[parfill]{parskip}    		% Activate to begin paragraphs with an empty line rather than an indent
\usepackage{graphicx}				% Use pdf, png, jpg, or eps§ with pdflatex; use eps in DVI mode
								% TeX will automatically convert eps --> pdf in pdflatex		
\usepackage{amssymb}

%SetFonts

%SetFonts


\title{Extending the Analysis: Supplement to ECF Chapter}
\author{Steven O. Kimbrough}
%\date{}							% Activate to display a given date or no date

\begin{document}

\newcount\draft
\draft=1

\maketitle
%\section{}
%\subsection{}



%\section{Extending the Analysis\label{ecfsec:extending_the_analysis}}


%The basic analysis of the previous section examined the comparative quality of the strategies employed by  role 1 players in a random realization of the game. It yielded an aliquot of insight, demonstrating that one player was optimal and one fared poorly in this single example. Comparison of the strategies employed by the several type 1 players suggests (in this one case of a random game) that indeed projects with high scores and high medians should be invested in, but that some spreading of resources among them is merited. But this is just one random game and one type or role of player. 

We now investigate the generalization to many random games and all five types or roles of players. For each of the five types of players, we randomly generate 1,000 game setups, instead of just one as in the previous section. We then examine how the individual strategies for each type of player would fare in the 1,000 setups and we report summary statistics. We proceed by type of player in the subsections that follow.

To solve the best response problem with formulate as a Knapsack problem as in the previous section and use the Bang-for-Buck\index{Bang-for-Buck} heuristic, implemented in Python. %See Appendix \ref{ch:bang-for-buck} for illustration of the heuristic in Excel.

%\newpage
 \begin{table}[h]
 \figtop
 
 \centering

\begin{tabular}{rrrrrrrrrrrrr}
Player & Type & P1 & P2 & P3 & P4 & P5 & P6 & P7 & P8 & P9 & P10 & Sum \\ \hline
0 & 1 & 0 & 0 & 0 & 0 & 0 & 0 & 30 & 20 & 20 & 10 & 80 \\ 
1 & 1 & 0 & 0 & 0 & 0 & 0 & 0 & 0 & 30 & 25 & 25 & 80 \\ 
2 & 1 & 0 & 0 & 0 & 0 & 0 & 25 & 20 & 20 & 10 & 5 & 80 \\ 
3 & 1 & 0 & 0 & 0 & 0 & 0 & 0 & 0 & 0 & 30 & 50 & 80 \\ 
\end{tabular}
 \caption{Strategies of all players type 1 for ECF Game 1.}
 \label{table:random_panel_game_1_type_1_players}
 \figbot
 \end{table}
 
 \begin{table}[h]
\figtop
\vskip 3 pt
\centering
\begin{tabular}{rrrrrrrrrrr}
 & P1 & P2 & P3 & P4 & P5 & P6 & P7 & P8 & P9 & P10 \\ 
  \cline{2-11} 
Player 1 & 5 & 15 & 25 & 35 & 45 & 55 & 65 & 75 & 85 & 95 \\ 
Player 2 & 95 & 5 & 15 & 25 & 35 & 45 & 55 & 65 & 75 & 85 \\ 
Player 3 & 85 & 95 & 5 & 15 & 25 & 35 & 45 & 55 & 65 & 75 \\ 
Player 4 & 75 & 85 & 95 & 5 & 15 & 25 & 35 & 45 & 55 & 65 \\ 
Player 5 & 65 & 75 & 85 & 95 & 5 & 15 & 25 & 35 & 45 & 55 \\  
 \cline{2-11} 
Sums & 325 & 275 & 225 & 175 & 125 & 175 & 225 & 275 & 325 & 375
\end{tabular}
\caption{Data set 1 for ECF Game 1.}
\label{table:ecf_game_1_data}
\figbot
\end{table}



\subsection{Focal Player: Role 1}

Table \ref{table:random_panel_game_1_type_1_players_2} reproduces Table \ref{table:random_panel_game_1_type_1_players}  with the addition of the project values for players in role 1 (presented in Table \ref{table:ecf_game_1_data}, page \pageref{table:ecf_game_1_data}). We see a general pattern of allocating thalers across the higher value projects, from player 1's perspective. Notice that player 3 is the most concentrated, dividing its thalers between projects P9 and P10. Player 2 is the most dispersed in its allocations.


 \begin{table}[h]
 \figtop
 
 \centering
\begin{tabular}{rrrrrrrrrrrrr}
 & Prefs: & 5 & 15 & 25 & 35 & 45 & 55 & 65 & 75 & 85 & 95 & \\  \hline
Player & Type & P1 & P2 & P3 & P4 & P5 & P6 & P7 & P8 & P9 & P10 & Sum \\ \hline
0 & 1 & 0 & 0 & 0 & 0 & 0 & 0 & 30 & 20 & 20 & 10 & 80 \\ 
1 & 1 & 0 & 0 & 0 & 0 & 0 & 0 & 0 & 30 & 25 & 25 & 80 \\ 
2 & 1 & 0 & 0 & 0 & 0 & 0 & 25 & 20 & 20 & 10 & 5 & 80 \\ 
3 & 1 & 0 & 0 & 0 & 0 & 0 & 0 & 0 & 0 & 30 & 50 & 80 \\ 

\end{tabular}
 \caption{Strategies of all players type 1 for ECF Game 1 (after Table \ref{table:random_panel_game_1_type_1_players}).}
 \label{table:random_panel_game_1_type_1_players_2}
 \figbot
\end{table}
Looking now at Table \ref{table:summary_game1_role1} we can see how the strategies of the four role 1 players fared during 1,000 randomly generated ECF games. As in our simple example from the previous section, Bang-for-Buck,\index{Bang-for-Buck} the optimal strategy (although estimated by a heuristic) does by far the best overall. Also as in the previous case, the strategy of player 3 does far better than any of the other player 1 strategies. In fact, the four strategies can be ranked (intuitively) by the degree to which they focus their thalers on projects P9 and P10. The ranking is $3 > 1> 0 > 2$, and this is exactly the performance order of the mean.
\begin{table}[h]
\figtop

\centering
\begin{tabular}{ccccccccc}
player & count & mean & std & min & 25\%\ & 50\%\ & 75\%\ & max \\ 
0 & 1000.0 & 22.015 & 35.1 & 0.0 & 0.0 & 0.0 & 15.0 & 100.0 \\ 
1 & 1000.0 & 49.355 & 51.5 & 0.0 & 0.0 & 15.0 & 95.0 & 185.0 \\ 
2 & 1000.0 & 10.85 & 24.6 & 0.0 & 0.0 & 0.0 & 15.0 & 100.0 \\ 
3 & 1000.0 & 111.69 & 53.0 & 0.0 & 95.0 & 100.0 & 180.0 & 195.0 \\ 
BfB & 1000.0 & 163.825 & 36.0 & 90.0 & 115.0 & 185.0 & 185.0 & 255.0 \\ \end{tabular}
\caption{Summary statistics for player role 1 strategies in ECF game 1.}
\label{table:summary_game1_role1}
\figbot
\end{table}

\newpage\clearpage

\subsection{Focal Player: Role 2}

Table \ref{table:random_panel_game_1_type_2_players} presents the strategies chosen by the
15 type 2 (role 2) players in the game, along with  the preference scores (project values) for each of the projects (from Table \ref{table:ecf_game_1_data}). The pattern in evidence is broadly that of the role 1 players, in Table \ref{table:random_panel_game_1_type_1_players_2}. Most of the thalers are allocated to high-value projects, but here there is considerably more spread in the disbursements.



 \begin{table}[h]
 \figtop
 
 \centering
\begin{tabular}{rrrrrrrrrrrrr}
 & Prefs: & 95 & 5 & 15 & 25 & 35 & 45 & 55 & 65 & 75 & 85 & \\\hline
Player & Type & P1 & P2 & P3 & P4 & P5 & P6 & P7 & P8 & P9 & P10 & Sum \\ \hline
4 & 2 & 20 & 0 & 0 & 0 & 0 & 0 & 0 & 20 & 20 & 20 & 80 \\ 
5 & 2 & 20 & 20 & 0 & 0 & 0 & 0 & 0 & 0 & 20 & 20 & 80 \\ 
6 & 2 & 15 & 20 & 0 & 0 & 0 & 0 & 0 & 0 & 30 & 15 & 80 \\ 
7 & 2 & 19 & 10 & 5 & 0 & 0 & 0 & 5 & 8 & 15 & 18 & 80 \\ 
8 & 2 & 34 & 0 & 0 & 0 & 0 & 0 & 0 & 0 & 12 & 34 & 80 \\ 
9 & 2 & 30 & 0 & 0 & 0 & 0 & 0 & 0 & 0 & 20 & 30 & 80 \\ 
10 & 2 & 30 & 0 & 0 & 0 & 0 & 0 & 0 & 0 & 25 & 25 & 80 \\ 
11 & 2 & 10 & 0 & 0 & 0 & 0 & 0 & 20 & 20 & 20 & 10 & 80 \\ 
12 & 2 & 30 & 1 & 0 & 0 & 0 & 0 & 0 & 4 & 20 & 25 & 80 \\ 
13 & 2 & 35 & 0 & 0 & 0 & 0 & 0 & 0 & 0 & 25 & 20 & 80 \\ 
14 & 2 & 30 & 0 & 0 & 0 & 0 & 0 & 0 & 10 & 20 & 20 & 80 \\ 
15 & 2 & 0 & 0 & 50 & 0 & 0 & 0 & 0 & 20 & 10 & 0 & 80 \\ 
16 & 2 & 30 & 0 & 0 & 0 & 0 & 0 & 0 & 0 & 20 & 30 & 80 \\ 
17 & 2 & 30 & 0 & 0 & 0 & 0 & 0 & 0 & 0 & 25 & 25 & 80 \\ 
18 & 2 & 16 & 0 & 0 & 0 & 8 & 8 & 8 & 8 & 16 & 16 & 80 \\ 
\end{tabular}
 \caption{Strategies of all players type 2 for ECF Game 1.}
 \label{table:random_panel_game_1_type_2_players}
 \figbot
\end{table}

\clearpage\newpage

Table \ref{table:summary_game1_role2} tells the story of how the various strategies fared. Bang-for-Buck\index{Band-for-Buck} of course does comparatively well. As expected, it does about as well as it did in the role of player 1. None of the player 2 strategies, however come close to the play 3 strategy in role 1, even though the payoffs for role 2 are a one-step rotation of the  payoffs for role 1. Notice that player 4 in Table \ref{table:summary_game1_role2} has the highest of the mean scores (71.45) among the players. This player corresponds to the 5th strategy in Table \ref{table:random_panel_game_1_type_2_players}, there labeled player 8. This strategy is arguably the most concentrated on the higher-value strategies and may be compared to 
strategy 1 in Table \ref{table:random_panel_game_1_type_1_players_2}, whose mean is 49.36 in Table \ref{table:summary_game1_role1}. This suggests a structural difference in the game between the two roles.
\begin{table}[h]
\figtop

\centering
\begin{tabular}{ccccccccc}
 player & count & mean & std & min & 25\%\ & 50\%\ & 75\%\ & max \\ 
0 & 1000.0 & 40.395 & 54.5 & 0.0 & 0.0 & 5.0 & 85.0 & 255.0 \\ 
1 & 1000.0 & 41.465 & 54.2 & 0.0 & 0.0 & 5.0 & 85.0 & 255.0 \\ 
2 & 1000.0 & 47.985 & 56.0 & 0.0 & 0.0 & 5.0 & 80.0 & 255.0 \\ 
3 & 1000.0 & 36.99 & 52.4 & 0.0 & 0.0 & 5.0 & 85.0 & 255.0 \\ 
4 & 1000.0 & 71.45 & 67.3 & 0.0 & 0.0 & 85.0 & 95.0 & 255.0 \\ 
5 & 1000.0 & 62.04 & 65.8 & 0.0 & 0.0 & 75.0 & 95.0 & 255.0 \\ 
6 & 1000.0 & 63.64 & 65.9 & 0.0 & 0.0 & 75.0 & 95.0 & 255.0 \\ 
7 & 1000.0 & 28.26 & 45.2 & 0.0 & 0.0 & 0.0 & 75.0 & 180.0 \\ 
8 & 1000.0 & 56.915 & 63.4 & 0.0 & 0.0 & 5.0 & 90.0 & 255.0 \\ 
9 & 1000.0 & 65.285 & 64.5 & 0.0 & 0.0 & 75.0 & 95.0 & 255.0 \\ 
10 & 1000.0 & 51.415 & 60.7 & 0.0 & 0.0 & 5.0 & 90.0 & 255.0 \\ 
11 & 1000.0 & 10.735 & 27.1 & 0.0 & 0.0 & 0.0 & 5.0 & 160.0 \\ 
12 & 1000.0 & 62.04 & 65.8 & 0.0 & 0.0 & 75.0 & 95.0 & 255.0 \\ 
13 & 1000.0 & 63.64 & 65.9 & 0.0 & 0.0 & 75.0 & 95.0 & 255.0 \\ 
14 & 1000.0 & 38.405 & 53.2 & 0.0 & 0.0 & 5.0 & 85.0 & 255.0 \\ 
BfB & 1000.0 & 173.765 & 52.4 & 80.0 & 160.0 & 170.0 & 185.0 & 260.0 \\ 
 \end{tabular}
\caption{Summary statistics for player role 2 strategies in ECF game 1.}
\label{table:summary_game1_role2}
\figbot
\end{table}

\newpage\clearpage

\subsection{Focal Player: Role 3}

Table \ref{table:random_panel_game_1_type_3_players} presents the strategies chosen by the
12 type 3 (role 3) players in the game, along with  the preference scores (project values) for each of the projects (from Table \ref{table:ecf_game_1_data}). The pattern in evidence is broadly that of the role 1 players, in Table \ref{table:random_panel_game_1_type_1_players_2}. Most of the thalers are allocated to high-value projects, but here there is considerably more spread in the disbursements.

 \begin{table}[h]
 \figtop
 
 \centering
\begin{tabular}{rrrrrrrrrrrrr}
 & Prefs: & 85 & 95 & 5 & 15 & 25 & 35 & 45 & 55 & 65 & 75 & \\ \hline
Player & Type & P1 & P2 & P3 & P4 & P5 & P6 & P7 & P8 & P9 & P10 & Sum \\ \hline
19 & 3 & 20 & 20 & 0 & 0 & 0 & 0 & 0 & 0 & 20 & 20 & 80 \\ 
20 & 3 & 32 & 0 & 0 & 0 & 0 & 0 & 0 & 0 & 24 & 24 & 80 \\ 
21 & 3 & 30 & 30 & 0 & 0 & 0 & 0 & 0 & 0 & 10 & 10 & 80 \\ 
22 & 3 & 25 & 40 & 0 & 0 & 0 & 0 & 0 & 0 & 0 & 15 & 80 \\ 
23 & 3 & 20 & 0 & 0 & 0 & 0 & 0 & 0 & 20 & 20 & 20 & 80 \\ 
24 & 3 & 30 & 0 & 0 & 0 & 0 & 0 & 0 & 0 & 20 & 30 & 80 \\ 
25 & 3 & 20 & 25 & 0 & 0 & 0 & 0 & 5 & 5 & 10 & 15 & 80 \\ 
26 & 3 & 30 & 50 & 0 & 0 & 0 & 0 & 0 & 0 & 0 & 0 & 80 \\ 
27 & 3 & 24 & 25 & 1 & 1 & 1 & 1 & 1 & 1 & 1 & 24 & 80 \\ 
28 & 3 & 20 & 20 & 0 & 0 & 0 & 0 & 0 & 0 & 20 & 20 & 80 \\ 
29 & 3 & 35 & 0 & 0 & 0 & 0 & 0 & 0 & 0 & 25 & 20 & 80 \\ 
30 & 3 & 0 & 0 & 0 & 0 & 0 & 0 & 0 & 20 & 30 & 30 & 80 \\  \end{tabular}
 \caption{Strategies of all players type 3 for ECF Game 1.}
 \label{table:random_panel_game_1_type_3_players}
 \figbot
\end{table}

\clearpage\newpage
Table \ref{table:summary_game1_role3} tells the story of how the various strategies fared. Bang-for-Buck,\index{Bang-for-Buck} of course, does comparatively well. As expected, it does about as well as it did in the roles of players 1 and 2. As in the case of player 2, none of the player 3 strategies come close to the player 3 strategy in role 1, even though the payoffs for role 3 are a two-step rotation of the  payoffs for role 1. Notice that player 7 in Table \ref{table:summary_game1_role3}  has the highest of the mean scores (74.45) among the players. This player corresponds to the 8th strategy in Table \ref{table:random_panel_game_1_type_3_players}, there labeled player ID 26. This strategy is arguably the most concentrated on the higher-value strategies and may be compared to 
strategy 1 in Table \ref{table:random_panel_game_1_type_1_players_2},  whose mean is 49.36 in Table \ref{table:summary_game1_role1}. This suggests a structural difference in the game between the two roles.

\begin{table}[h]
\figtop

\centering
\begin{tabular}{ccccccccc}
 player & count & mean & std & min & 25\%\ & 50\%\ & 75\%\ & max \\ 
0 & 1000.00 & 59.40 & 58.84 & 0.00 & 0.00 & 65.00 & 95.00 & 235.00 \\ 
1 & 1000.00 & 72.33 & 64.43 & 0.00 & 0.00 & 65.00 & 140.00 & 235.00 \\ 
2 & 1000.00 & 59.90 & 56.63 & 0.00 & 0.00 & 75.00 & 95.00 & 255.00 \\ 
3 & 1000.00 & 68.04 & 57.44 & 0.00 & 0.00 & 85.00 & 95.00 & 255.00 \\ 
4 & 1000.00 & 45.81 & 53.72 & 0.00 & 0.00 & 0.00 & 75.00 & 235.00 \\ 
5 & 1000.00 & 64.50 & 62.40 & 0.00 & 0.00 & 65.00 & 95.00 & 235.00 \\ 
6 & 1000.00 & 52.86 & 55.12 & 0.00 & 0.00 & 65.00 & 95.00 & 235.00 \\ 
7 & 1000.00 & 74.45 & 54.12 & 0.00 & 0.00 & 95.00 & 95.00 & 255.00 \\ 
8 & 1000.00 & 62.50 & 59.45 & 0.00 & 0.00 & 75.00 & 95.00 & 235.00 \\ 
9 & 1000.00 & 59.40 & 58.84 & 0.00 & 0.00 & 65.00 & 95.00 & 235.00 \\ 
10 & 1000.00 & 68.47 & 62.66 & 0.00 & 0.00 & 65.00 & 95.00 & 235.00 \\ 
11 & 1000.00 & 61.83 & 52.77 & 0.00 & 0.00 & 65.00 & 75.00 & 235.00 \\ 
BfB & 1000.00 & 188.93 & 47.78 & 65.00 & 160.00 & 180.00 & 235.00 & 320.00 \\ 
 \end{tabular}
\caption{Summary statistics: player role 3 strategies in ECF game 1.}
\label{table:summary_game1_role3}
\figbot
\end{table}



\newpage\clearpage
\subsection{Focal Player: Role 4}

Table \ref{table:random_panel_game_1_type_4_players} presents the strategies chosen by the
16 type 4 (role 4) players in the game, along with  the preference scores (project values) for each of the projects (from Table \ref{table:ecf_game_1_data}). The pattern in evidence is broadly that of the role 1 players, in Table \ref{table:random_panel_game_1_type_1_players_2}. Most of the thalers are allocated to high-value projects, but here there is considerably more spread in the disbursements.


 \begin{table}[h]
 \figtop
 
 \centering
\begin{tabular}{rrrrrrrrrrrrr}
& Prefs: & 75 & 85 & 95 & 5 & 15 & 25 & 35 & 45 & 55 & 65 & \\ \hline
Player & Type & P1 & P2 & P3 & P4 & P5 & P6 & P7 & P8 & P9 & P10 & Sum \\ \hline
31 & 4 & 10 & 50 & 0 & 0 & 0 & 0 & 0 & 0 & 10 & 10 & 80 \\ 
32 & 4 & 15 & 50 & 0 & 0 & 0 & 0 & 0 & 0 & 0 & 15 & 80 \\ 
33 & 4 & 45 & 0 & 0 & 0 & 0 & 0 & 0 & 0 & 0 & 35 & 80 \\ 
34 & 4 & 35 & 1 & 0 & 0 & 1 & 1 & 1 & 1 & 10 & 30 & 80 \\ 
35 & 4 & 20 & 8 & 5 & 0 & 0 & 0 & 5 & 7 & 15 & 20 & 80 \\ 
36 & 4 & 25 & 0 & 0 & 0 & 0 & 0 & 0 & 15 & 20 & 20 & 80 \\ 
37 & 4 & 20 & 0 & 0 & 0 & 0 & 0 & 0 & 20 & 20 & 20 & 80 \\ 
38 & 4 & 20 & 20 & 20 & 0 & 0 & 0 & 0 & 0 & 0 & 20 & 80 \\ 
39 & 4 & 30 & 0 & 0 & 0 & 0 & 0 & 0 & 20 & 10 & 20 & 80 \\ 
40 & 4 & 15 & 40 & 0 & 0 & 0 & 0 & 0 & 0 & 10 & 15 & 80 \\ 
41 & 4 & 35 & 0 & 0 & 0 & 0 & 0 & 0 & 0 & 35 & 10 & 80 \\ 
42 & 4 & 0 & 0 & 40 & 0 & 0 & 0 & 0 & 0 & 20 & 20 & 80 \\ 
43 & 4 & 20 & 30 & 0 & 0 & 0 & 0 & 0 & 10 & 5 & 15 & 80 \\ 
44 & 4 & 40 & 0 & 0 & 0 & 0 & 0 & 0 & 0 & 0 & 40 & 80 \\ 
45 & 4 & 20 & 20 & 20 & 0 & 0 & 0 & 0 & 0 & 0 & 20 & 80 \\ 
46 & 4 & 15 & 15 & 30 & 0 & 0 & 0 & 0 & 0 & 10 & 10 & 80 \\ 
\end{tabular}
 \caption{Strategies of all players type 4 for ECF Game 1.}
 \label{table:random_panel_game_1_type_4_players}
 \figbot
\end{table}

\newpage\clearpage

Table \ref{table:summary_game1_role4} tells the story of how the various strategies fared. Bang-for-Buck,\index{Bang-for-Buck} of course, does comparatively well. As expected, it does about as well as it did in the roles of players 1 and 2. As in the case of player 2, none of the player 3 strategies come close to the player 3 strategy in role 1, even though the payoffs for role 3 are a two-step rotation of the  payoffs for role 1. Notice that player 7 in Table \ref{table:summary_game1_role3}  has the highest of the mean scores (74.45) among the players. This player corresponds to the 8th strategy in Table \ref{table:random_panel_game_1_type_3_players}, there labeled player ID 26. This strategy is arguably the most concentrated on the higher-value strategies and may be compared to 
strategy 1 in Table \ref{table:random_panel_game_1_type_1_players_2},  whose mean is 49.36 in Table \ref{table:summary_game1_role1}. This suggests a structural difference in the game between the two roles.


\begin{table}[h]
\figtop

\centering
\begin{tabular}{ccccccccc}
 player & count & mean & std & min & 25\%\ & 50\%\ & 75\%\ & max \\ 
0 & 1000.00 & 72.05 & 48.20 & 0.00 & 55.00 & 85.00 & 85.00 & 205.00 \\ 
1 & 1000.00 & 68.47 & 48.00 & 0.00 & 0.00 & 85.00 & 85.00 & 205.00 \\ 
2 & 1000.00 & 75.08 & 54.88 & 0.00 & 0.00 & 75.00 & 140.00 & 225.00 \\ 
3 & 1000.00 & 60.91 & 55.48 & 0.00 & 0.00 & 65.00 & 85.00 & 225.00 \\ 
4 & 1000.00 & 40.06 & 46.67 & 0.00 & 0.00 & 0.00 & 65.00 & 205.00 \\ 
5 & 1000.00 & 42.53 & 47.43 & 0.00 & 0.00 & 55.00 & 65.00 & 225.00 \\ 
6 & 1000.00 & 38.78 & 44.55 & 0.00 & 0.00 & 0.00 & 65.00 & 150.00 \\ 
7 & 1000.00 & 36.91 & 45.87 & 0.00 & 0.00 & 0.00 & 65.00 & 205.00 \\ 
8 & 1000.00 & 38.91 & 48.77 & 0.00 & 0.00 & 0.00 & 65.00 & 225.00 \\ 
9 & 1000.00 & 65.16 & 51.38 & 0.00 & 0.00 & 85.00 & 85.00 & 205.00 \\ 
10 & 1000.00 & 68.16 & 47.55 & 0.00 & 55.00 & 55.00 & 75.00 & 195.00 \\ 
11 & 1000.00 & 35.71 & 42.52 & 0.00 & 0.00 & 0.00 & 65.00 & 150.00 \\ 
12 & 1000.00 & 52.85 & 49.88 & 0.00 & 0.00 & 65.00 & 85.00 & 225.00 \\ 
13 & 1000.00 & 69.36 & 54.64 & 0.00 & 0.00 & 65.00 & 140.00 & 225.00 \\ 
14 & 1000.00 & 36.91 & 45.87 & 0.00 & 0.00 & 0.00 & 65.00 & 205.00 \\ 
15 & 1000.00 & 33.59 & 43.58 & 0.00 & 0.00 & 0.00 & 65.00 & 150.00 \\ 
BfB & 1000.00 & 173.85 & 38.40 & 75.00 & 140.00 & 160.00 & 205.00 & 280.00 \\ 
 \end{tabular}
\caption{Summary statistics: player role 4 strategies in ECF game 1.}
\label{table:summary_game1_role4}
\figbot
\end{table}

\newpage\clearpage
\subsection{Focal Player: Role 5}

Table \ref{table:random_panel_game_1_type_5_players} presents the strategies chosen by the
3 type 5 (role 5) players in the game, along with  the preference scores (project values) for each of the projects (from Table \ref{table:ecf_game_1_data}). The pattern in evidence is only very broadly that of the role 1 players, in Table \ref{table:random_panel_game_1_type_1_players_2}. 
The highest value projects for role 5 are P2, P3, P4, but there is no investment at all in the two highest value projects P3 and P4. Instead the players focus on the two ends of the project list, P1--P2 and P9--P10.
 \begin{table}[h]
 \figtop
 
 \centering
\begin{tabular}{rrrrrrrrrrrrr}
& Prefs: & 65 & 75 & 85 & 95 & 5 & 15 & 25 & 35 & 45 & 55 \\ \hline
Player & Type & P1 & P2 & P3 & P4 & P5 & P6 & P7 & P8 & P9 & P10 & Sum \\ \hline
47 & 5 & 10 & 30 & 0 & 0 & 0 & 0 & 0 & 10 & 20 & 10 & 80 \\ 
48 & 5 & 20 & 50 & 0 & 0 & 0 & 0 & 0 & 0 & 0 & 10 & 80 \\ 
49 & 5 & 0 & 50 & 0 & 0 & 0 & 0 & 0 & 0 & 30 & 0 & 80 \\ 
\end{tabular}
 \caption{Strategies of all players type 5 for ECF Game 1.}
 \label{table:random_panel_game_1_type_5_players}
 \figbot
\end{table}

Table \ref{table:summary_game1_role5} tells the story of how the various strategies fared. Bang-for-Buck,\index{Bang-for-Buck} of course, does  well. Because no player really focused on its high-value projects all players do comparatively poorly.

% As expected, it does about as well as it did in the roles of players 1 and 2. As in the case of player 2, none of the player 3 strategies come close to the player 3 strategy in role 1, even though the payoffs for role 3 are a two-step rotation of the  payoffs for role 1. Notice that player 7 in Table \ref{table:summary_game1_role3}  has the highest of the mean scores (74.45) among the players. This player corresponds to the 8th strategy in Table \ref{table:random_panel_game_1_type_3_players}, there labeled player ID 26. This strategy is arguably the most concentrated on the higher-value strategies and may be compared to 
%strategy 1 in Table \ref{table:random_panel_game_1_type_1_players_2},  whose mean is 49.36 in Table \ref{table:summary_game1_role1}. This suggests a structural difference in the game between the two roles.
\begin{table}[h]
\figtop

\centering
\begin{tabular}{ccccccccc}
 player & count & mean & std & min & 25\%\ & 50\%\ & 75\%\ & max \\ 
0 & 1000.00 & 38.06 & 43.31 & 0.00 & 0.00 & 22.50 & 65.00 & 165.00 \\ 
1 & 1000.00 & 49.88 & 45.81 & 0.00 & 0.00 & 55.00 & 75.00 & 195.00 \\ 
2 & 1000.00 & 45.59 & 41.53 & 0.00 & 0.00 & 45.00 & 75.00 & 195.00 \\ 
BfB & 1000.00 & 148.32 & 30.54 & 55.00 & 120.00 & 165.00 & 165.00 & 240.00 \\ 
 \end{tabular}
\caption{Summary statistics: player role 5 strategies in ECF game 1.}
\label{table:summary_game1_role5}
\figbot
\end{table}

\newpage\clearpage


\ifnum\draft=1
\vfill
\noindent File: Extending\_the\_analysis.tex
\fi


\end{document}  