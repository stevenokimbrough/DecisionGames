%There are various examples in the fancyhdr documentation of redefining the \headrulewidth:
%
%\fancyhf{} % sets both header and footer to nothing
%\renewcommand{\headrulewidth}{0pt}
%% your new footer definitions here
%will do what you want.
%
%To learn more about fancyhdr, you could either look into its documentation or at the respective chapter in the Wikibook LaTeX.

\documentclass[11pt, oneside]{article}   	% use "amsart" instead of "article" for AMSLaTeX format
\usepackage[margin=1in]{geometry}                		% See geometry.pdf to learn the layout options. There are lots.
\geometry{letterpaper}                   		% ... or a4paper or a5paper or ... 
%\geometry{landscape}                		% Activate for for rotated page geometry
%\usepackage[parfill]{parskip}    		% Activate to begin paragraphs with an empty line rather than an indent
\usepackage{graphicx}				% Use pdf, png, jpg, or eps� with pdflatex; use eps in DVI mode
								% TeX will automatically convert eps --> pdf in pdflatex		
\usepackage{amssymb}

%%%%%%%%%%%%%%%%%%%%%%%%
\usepackage{mathptmx}
\usepackage{anyfontsize}
\usepackage{t1enc}
%SetFonts

%SetFonts

\usepackage[figuresright]{rotating} % for sideways and sidewaystable, etc. environments
% instead, using: see http://www.tex.ac.uk/cgi-bin/texfaq2html?label=landscape
%
\usepackage{rotfloat}
% the following package, when present, lets the figures and tables orient oppositely on even and odd pages
%
\usepackage{lscape}

\usepackage{fancyhdr}
\usepackage{natbib}
\usepackage{hyperref}
%%%%%%%%%%%%%%%%%%%%%%%%%%%%%%

\title{Stylized Redistricting Game, Part 4}
\author{Steven O. Kimbrough \and Peter Miller}
%\date{}							% Activate to display a given date or no date

\begin{document}
\pagestyle{fancy}
\renewcommand{\headrulewidth}{0pt}
\maketitle
%\section{}
%\subsection{}

%This exercise in stylized redistricting  is inspired by a report by \cite{doherty_ryan_2014}. 
We will work with a stylized polity map, shown on page 
  \pageref{polity_map}. In this polity there are $8\times 8=64$ basic areal units, called blocks or cells. Assume that each block is equal in population to the others. The blocks are marked D or R, for whether there are Democratic party or Republican party majorities in them. As you can verify, there are 34 Democratic party majority blocks and 30 Republican party majority blocks. 
  
  You are to come up with a legally valid redistricting plan, consisting of 8 districts of 8 contiguous blocks each. Point contiguity is permitted.
  In addition to the partisan makeup of the blocks, we are concerned in this exercise with communities of interest. In particular, there are six blocks present having a majority of a protected community of interest. Think: African Americans and the Voting Rights Act. These blocks are indicated with {\textit{\textbf{italic bold face type}}} and are all blocks with Democratic party majorities.
  Further, your team has been assigned to propose a redistricting plan to a neutral, non-partisan commission, which will do its best to choose a plan that is in the public interest and that is fair to this identified community of interest.
  
%Using the Excel file {\it HandoutMap.xlsx} as the source for the polity map, you should indicate your 8 districts clearly on a copy of the {\it HandoutMap.xlsx} file. You might use coloring to indicate the districts, or outlining (bolding the borders, e.g.). You can also use the coordinate naming scheme described below. Whatever you do please include a key so your choices are clear. Finally, your team should write up a 400 word  essay (no longer, shorter is OK) arguing why your proposal is indeed valid and fair. Hand everything (Excel file plus essay file) in on Canvas.
  You should draw your 8 districts clearly on the polity map on page  \pageref{polity_map} and hand in this exercise booklet with the requested information below supplied. Page 
   \pageref{practice_map} contains a copy of the polity map on page  \pageref{polity_map}. You can use it for practice, before finalizing the districts you draw on the polity map on page  \pageref{polity_map}. Feel free to remove it from the booklet and not hand it in.
   \vskip 6 pt
  \centerline{* * *}
  \vskip 6 pt
%  \noindent To which type of polity has your team been assigned? (Circle your answer.)
%  \vskip 15 pt
%  \noindent \flushleft{Controlled by the Democratic party.}\hfill{Controlled by the Republican party.}
%  \vskip 6 pt
%  \centerline{* * *}
%  \vskip 6 pt
  \noindent Print your team member names here: 
\vskip 25 pt
\noindent\rule{\textwidth}{0.05cm} % {12cm}
  \vskip 6 pt
  \centerline{* * *}
  \vskip 6 pt
  \noindent What is the composition of your plan?
 \begin{itemize}
 \item Number of Democratic majority districts: \rule{1in}{0.05cm}
 \item Number of Republican majority districts: 
  \rule{1in}{0.05cm}
  \item Number of party balanced districts:  \rule{1in}{0.05cm}
  \end{itemize}
  
  \vfill
  \noindent File: StylizedRedistricting4inclass.tex
\newpage\clearpage
\rfoot{Polity map}
\lfoot{Polity map}%\label{polity_map}
% up: 63, right 75
\begin{sidewaysfigure}
\begin{picture}(600,504)(0,0) % 75, 63 for 8. up 63, across 75.
\put(0,0){\framebox(600,504)}
\put(0,63){\line(1,0){600}}
\put(0,126){\line(1,0){600}}
\put(0,189){\line(1,0){600}}
\put(0,252){\line(1,0){600}}
\put(0,315){\line(1,0){600}}
\put(0,378){\line(1,0){600}}
\put(0,441){\line(1,0){600}}
% now the vertical lines
\put(75,0){\line(0,1){504}}
\put(150,0){\line(0,1){504}}
\put(225,0){\line(0,1){504}}
\put(300,0){\line(0,1){504}}
\put(375,0){\line(0,1){504}}
\put(450,0){\line(0,1){504}}
\put(525,0){\line(0,1){504}}

% Now the letters

% row 2 from bottom:
\multiput(18,77)(75,0){5}{{\fontsize{50}{60} \selectfont R}}
\multiput(393,77)(75,0){3}{{\fontsize{50}{60} \selectfont D}}
%
\multiput(18,140)(0,63){6}{{\fontsize{50}{60} \selectfont D}}
% bottom row:
\multiput(18,14)(75,0){3}{{\fontsize{50}{60} \selectfont D}}
\multiput(243,14)(75,0){3}{{\fontsize{50}{60} \selectfont R}}
\multiput(468,14)(75,0){2}{{\fontsize{50}{60} \selectfont D}}
% row 3 after first column:
\multiput(93,140)(75,0){2}{{\fontsize{50}{60} \selectfont D}}
\multiput(243,140)(75,0){1}{{\fontsize{50}{60} \selectfont R}}
\multiput(318,140)(75,0){4}{{\fontsize{50}{60} \selectfont D}}
% row 4 after first column:
\multiput(93,203)(75,0){2}{{\fontsize{50}{60} \selectfont D}}
\multiput(243,203)(75,0){1}{{\fontsize{50}{60} \selectfont R}}
\multiput(318,203)(75,0){4}{{\fontsize{50}{60} \selectfont {\textit{\textbf D}} }} %%%% African American
% row 5 after first column:
\multiput(93,266)(75,0){3}{{\fontsize{50}{60} \selectfont R}}
\multiput(318,266)(75,0){2}{{\fontsize{50}{60} \selectfont {\textit{\textbf D}} }} %%%% African American
\multiput(468,266)(75,0){2}{{\fontsize{50}{60} \selectfont R}}
% row 6 after first column:
\multiput(93,329)(75,0){1}{{\fontsize{50}{60} \selectfont R}}
\multiput(168,329)(75,0){1}{{\fontsize{50}{60} \selectfont D}}
\multiput(243,329)(75,0){2}{{\fontsize{50}{60} \selectfont R}}
\multiput(393,329)(75,0){1}{{\fontsize{50}{60} \selectfont D}}
\multiput(468,329)(75,0){2}{{\fontsize{50}{60} \selectfont R}}
% row 7 after first column:
\multiput(93,392)(75,0){1}{{\fontsize{50}{60} \selectfont R}}
\multiput(168,392)(75,0){1}{{\fontsize{50}{60} \selectfont D}}
\multiput(243,392)(75,0){2}{{\fontsize{50}{60} \selectfont R}}
\multiput(393,392)(75,0){1}{{\fontsize{50}{60} \selectfont D}}
\multiput(468,392)(75,0){2}{{\fontsize{50}{60} \selectfont R}}
% row 8 after first column:
\multiput(93,455)(75,0){2}{{\fontsize{50}{60} \selectfont D}}
\multiput(243,455)(75,0){5}{{\fontsize{50}{60} \selectfont R}}
\end{picture}
%\caption{999}
\label{polity_map}
\end{sidewaysfigure}

\newpage\clearpage
%\thispagestyle{plain}
Note: We can refer to the blocks with a 2-D coordinate scheme. Let the lower left-hand corner block be (8,1), the upper right-hand corner block be (1,8), the upper left-hand block be (1,1), the lower right-hand block be (8,8), and the other blocks identified similarly.

%\bibliographystyle{apalike}
%\bibliography{union}
%\end{document}
\newpage\clearpage

\thispagestyle{plain}
\rfoot{Practice sheet}
\lfoot{Practice sheet}

%\newpage
% up: 63, right 75
\begin{sidewaysfigure}
\begin{picture}(600,504)(0,0) % 75, 63 for 8. up 63, across 75.
\put(0,0){\framebox(600,504)}
\put(0,63){\line(1,0){600}}
\put(0,126){\line(1,0){600}}
\put(0,189){\line(1,0){600}}
\put(0,252){\line(1,0){600}}
\put(0,315){\line(1,0){600}}
\put(0,378){\line(1,0){600}}
\put(0,441){\line(1,0){600}}
% now the vertical lines
\put(75,0){\line(0,1){504}}
\put(150,0){\line(0,1){504}}
\put(225,0){\line(0,1){504}}
\put(300,0){\line(0,1){504}}
\put(375,0){\line(0,1){504}}
\put(450,0){\line(0,1){504}}
\put(525,0){\line(0,1){504}}

% Now the letters

% row 2 from bottom:
\multiput(18,77)(75,0){5}{{\fontsize{50}{60} \selectfont R}}
\multiput(393,77)(75,0){3}{{\fontsize{50}{60} \selectfont D}}
%
\multiput(18,140)(0,63){6}{{\fontsize{50}{60} \selectfont D}}
% bottom row:
\multiput(18,14)(75,0){3}{{\fontsize{50}{60} \selectfont D}}
\multiput(243,14)(75,0){2}{{\fontsize{50}{60} \selectfont R}}
\multiput(393,14)(75,0){3}{{\fontsize{50}{60} \selectfont D}}
% row 3 after first column:
\multiput(93,140)(75,0){2}{{\fontsize{50}{60} \selectfont D}}
\multiput(243,140)(75,0){1}{{\fontsize{50}{60} \selectfont R}}
\multiput(318,140)(75,0){4}{{\fontsize{50}{60} \selectfont D}}
% row 4 after first column:
\multiput(93,203)(75,0){2}{{\fontsize{50}{60} \selectfont D}}
\multiput(243,203)(75,0){1}{{\fontsize{50}{60} \selectfont R}}
\multiput(318,203)(75,0){4}{{\fontsize{50}{60} \selectfont {\textit{\textbf D}} }}  %%%% African American
% row 5 after first column:
\multiput(93,266)(75,0){3}{{\fontsize{50}{60} \selectfont R}}
\multiput(318,266)(75,0){2}{{\fontsize{50}{60} \selectfont {\textit{\textbf D}} }} %%%% African American
\multiput(468,266)(75,0){2}{{\fontsize{50}{60} \selectfont R}}
% row 6 after first column:
\multiput(93,329)(75,0){1}{{\fontsize{50}{60} \selectfont R}}
\multiput(168,329)(75,0){1}{{\fontsize{50}{60} \selectfont D}}
\multiput(243,329)(75,0){2}{{\fontsize{50}{60} \selectfont R}}
\multiput(393,329)(75,0){1}{{\fontsize{50}{60} \selectfont D}}
\multiput(468,329)(75,0){2}{{\fontsize{50}{60} \selectfont R}}
% row 7 after first column:
\multiput(93,392)(75,0){1}{{\fontsize{50}{60} \selectfont R}}
\multiput(168,392)(75,0){1}{{\fontsize{50}{60} \selectfont D}}
\multiput(243,392)(75,0){2}{{\fontsize{50}{60} \selectfont R}}
\multiput(393,392)(75,0){1}{{\fontsize{50}{60} \selectfont D}}
\multiput(468,392)(75,0){2}{{\fontsize{50}{60} \selectfont R}}
% row 8 after first column:
\multiput(93,455)(75,0){2}{{\fontsize{50}{60} \selectfont D}}
\multiput(243,455)(75,0){5}{{\fontsize{50}{60} \selectfont R}}
\end{picture}
%\caption{999}
\label{practice_map}
\end{sidewaysfigure}
%Practice sheet.
%
\end{document}  
